%**********************
% Document type switch
%**********************
%\def\preprint{0}
\def\preprint{1}

%****************
% Document class
%****************
%\documentclass[UTF8]{article}
\documentclass{article}

%****************
% Packages, etc.
%****************
\usepackage{ifthen}
\ifthenelse{\preprint = 1}{
  \usepackage{amsthm,fullpage}
  \newtheorem{theorem}{Theorem}[section]
  \newtheorem{lemma}[theorem]{Lemma}
  \newtheorem{corollary}[theorem]{Corollary}
}{
%\usepackage{tocloft}
}
\usepackage[normalem]{ulem}

%**********
% Packages
%**********
\usepackage{amsmath,amsfonts,amssymb,hyperref,multirow}
%\usepackage[noend]{algpseudocode}
\usepackage{algpseudocode}
\usepackage{xcolor}
\hypersetup{colorlinks}
\newcommand{\algnote}[1]{\footnotesize \sc{Note: \it#1 } }

\usepackage{amssymb}
\usepackage{lastpage}
% for chinese
% \usepackage{ctex}
% to include pdf/eps/png files
\usepackage{graphicx}
% useful to add 'todo' markers
\usepackage{todonotes}
% hyperrefs
\usepackage{hyperref}
% nice source code formatting
\usepackage{minted}
% change style of section headings
\usepackage{sectsty}
\allsectionsfont{\sffamily}
% only required for orgmode ticked TODO items, can remove
\usepackage{amssymb}
% only required for underlining text
\usepackage[normalem]{ulem}

% often use this in differential operators:
\renewcommand{\d}{\ensuremath{\mathrm{d}}}

%% allow more reasonable text width for most documents than LaTeX default
%\setlength{\textheight}{21cm}
%\setlength{\textwidth}{16cm}
%
%% reduce left and right margins accordingly
%\setlength{\evensidemargin}{-0cm}
%\setlength{\oddsidemargin}{-0cm}
%
%% reduce top margin
%\setlength{\topmargin}{0cm}

% Increase default line spacing a little if desired
\renewcommand{\baselinestretch}{1.2}

% tailored float handling
%\renewcommand{\topfraction}{0.8}
%\renewcommand{\bottomfraction}{0.6}
%\renewcommand{\textfraction}{0.2}

%*******
% Title
%*******
\title{\sffamily \textbf{Title -- template for orgmode latex production}}

%*********
% Authors
%*********
\author{Xu Zhenkai}

%**********
% Document
%**********
\begin{document}

%*******
% Title
%*******
\maketitle

%**********
% Abstract
%**********
\begin{abstract}
  This is an abstract abstract, in the sense of only providing a
  virtual abstract, also known as the interface. Somebody will have to
  provide an inherited class that provides the real abstract.

  We have placed the abstract in the paper.tex file, so that all the
  content in the orgmode file \texttt{content.org} is organised into
  sections, and they can be unfolded, re-arranged, etc (the abstract
  doesn't go well into a section because it appears even before the
  first section starts). If you prefer, you can move the abstract
  latex definition as is into the \texttt{content.org} file, and all
  will work as before.
\end{abstract}

\ifthenelse{\preprint = 0}{

%**********
% Keywords
%**********
\begin{keywords}
    TODO: Add Keywords
\end{keywords}

%*****
% AMS
%*****
\begin{AMS}
  65K05, 68Q25, 68T05, 90C06, 90C30, 90C90
\end{AMS}

}{}

%*******************
% Table of Contents
%*******************
\ifthenelse{\preprint = 1}{
  %\newpage
  \tableofcontents
  \newpage
}{}
% include body of the paper, auto generated from orgmode content.org file
\input{content.tex}


\bibliographystyle{abbrv}
\bibliography{paper.bib}

\end{document}
%%% Local Variables:
%%% mode: latex
%%% TeX-command-extra-options: "-shell-escape"
%%% TeX-master: t
%%% TeX-engine: xetex
%%% End:

